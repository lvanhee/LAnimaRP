\documentclass[paper=a4, fontsize=11pt]{scrartcl}
%\usepackage[T1]{fontenc}
\usepackage{fourier}
\usepackage[T1]{fontenc}
\usepackage[latin1]{inputenc}
\usepackage[cyr]{aeguill}
\usepackage[francais]{babel}													% English language/hyphenation
\usepackage[protrusion=true,expansion=true]{microtype}	
\usepackage{amsmath,amsfonts,amsthm} % Math packages
\usepackage[pdftex]{graphicx}	
\usepackage{url}
\usepackage{verbatim}

%%% Custom sectioning
\usepackage{sectsty}
\allsectionsfont{\centering \normalfont\scshape}


%%% Custom headers/footers (fancyhdr package)
\usepackage{fancyhdr}
\pagestyle{fancyplain}
\fancyhead{}											% No page header
\fancyfoot[L]{}											% Empty 
\fancyfoot[C]{}											% Empty
\fancyfoot[R]{\thepage}									% Pagenumbering
\renewcommand{\headrulewidth}{0pt}			% Remove header underlines
\renewcommand{\footrulewidth}{0pt}				% Remove footer underlines
\setlength{\headheight}{13.6pt}

%%% Equation and float numbering
\numberwithin{equation}{section}		% Equationnumbering: section.eq#
\numberwithin{figure}{section}			% Figurenumbering: section.fig#
\numberwithin{table}{section}				% Tablenumbering: section.tab#


%%% Maketitle metadata
\newcommand{\horrule}[1]{\rule{\linewidth}{#1}} 	% Horizontal rule

\title{
		%\vspace{-1in} 	
		\usefont{OT1}{bch}{b}{n}
		\normalfont \normalsize \textsc{LAnimaRP -- Licence d'utilisation} \\ [25pt]
		\horrule{0.5pt} \\[0.4cm]
		\huge {Licence d'utilisation de LAnimaRP} \\
		\horrule{2pt} \\[0.5cm]
}
\author{
		\normalfont
        Lo�s Vanh�e\\[-3pt]		\normalsize
}

\begin{document}
\maketitle

Ce logiciel est librement distribu� sans compensation mon�taire de votre part. Cependant, pour qu'il continue � s'am�liorer et � �tre utile pour la communaut� GNiste, j'ai quelques demandes � vous soumettre pour utiliser ce logiciel. Point d'argent ni d'avantages en nature, mais plut�t de quoi am�liorer cet outil pour la suite afin que de futurs utilisateurs puissent, comme vous aujourd'hui, profiter de ce logiciel. Je vous fais totalement confiance pour bien vouloir acc�der � ces requ�te, en esp�rant ne pas avoir � changer cette politique � l'avenir. Ainsi, si vous utilisez ce logiciel, je vous demande cordialement de :
\begin{enumerate}
\item partager vos animations (fichiers de configuration et de ressource) de sorte que d'autres puissent les utiliser � leur tour (envoyez moi vos fichiers je m'occupe de la publication). Des clauses de confidentialit� temporaires sont n�gociables.
\item �crivez moi une documentation de son utilisation. Rien d'�norme, quelques lignes. Indiquez ce � quoi il sert pour le jeu, comment il est utilis�, s'il a �t� utile, s'il y a des pistes d'am�liorations, quelques photos de l'animation ``en contexte". Les �l�ments pertinents
\item un peu de publicit�, car un outil n'a aucune valeur s'il n'est pas connu. L� aussi, rien d'�norme : une mention du logiciel dans les docs de jeu ``techniques" et une mention pendant le d�brief final.
\item si on se croise, venez m'en parler.  Votre feedback direct est une source d'information primordiale pour am�liorer le logiciel.
\item optionnel : donnez-moi un souvenir physique de petite taille (petit mot, goodies, objet de jeu qui ne servira plus). J'ai une collection d'objets pour me souvenir que mes contributions ont aid� � un projet que j'aime remplir.
\end{enumerate}

En retour de quoi, je vous offre mon soutien. Tout d'abord, par un libre acc�s au logiciel. Ensuite, en faisant de la publicit� pour votre jeu : la collaboration sera pr�sent�e sur les m�dias relatifs � LAnimaRP, des vid�os pr�sentant les animations de votre jeu, etc. Enfin, en �tendant les fonctionnalit�s pour contribuer � vos besoins particuliers (plus de simulations, connexions � d'autres services etc).


\end{document}